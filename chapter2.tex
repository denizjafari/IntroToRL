\documentclass[answers]{exam}
\usepackage{array}
\usepackage{tabularx}
\usepackage{multirow}
\usepackage{longtable}
\usepackage{seqsplit}

% BASE TAKEN FROM ICCS315 SCRIBE NOTES

% --- SETUP STUFF ---
\usepackage[a4paper, margin=1in]{geometry}
\usepackage{enumitem}
\usepackage{booktabs}

\usepackage{url}
\usepackage[unicode]{hyperref}

\setcounter{secnumdepth}{2}

\usepackage{titlesec,blindtext,color}

% --- MATH STUFF ---
\usepackage{amsthm, amsmath, amssymb}
\usepackage{mathtools,xspace}
\usepackage{nicefrac}

\usepackage{bbm}
\usepackage{dsfont}
\usepackage{cancel}

\usepackage{blkarray}
\newcommand{\matindex}[1]{\mbox{\scriptsize#1}} % Matrix index

% --- FONT STUFF ---
% Has to be under math stuff for some reason :/
\usepackage{newpxtext, newpxmath}
\usepackage[T1]{fontenc}

% --- DIAGRAM STUFF ---
\usepackage{tikz,pgfplots,xcolor,graphicx}
\usepackage{graphicx}
\usepackage{tabularx}
\usepackage{colortbl}
\usepackage{caption}
\usepackage{subcaption}

\usepackage[breakable,skins]{tcolorbox}
\usepackage{framed}
\usepackage{mdframed}
\usepackage{float}

\pgfplotsset{compat=1.18}

% --- THEOREM STUFF ---
\newtheorem{theorem}{Theorem}[section]
\newtheorem{proposition}[theorem]{Proposition}
\newtheorem{lemma}[theorem]{Lemma}
\newtheorem{corollary}[theorem]{Corollary}

\newtheorem{exercise}[theorem]{Exercise}

\theoremstyle{definition}
\newtheorem{definition}[theorem]{Definition}
\newtheorem{example}[theorem]{Example}

\theoremstyle{remark}
\newtheorem{remark}[theorem]{Remark}
\newtheorem{claim}[theorem]{Claim}
\newtheorem{fact}[theorem]{Fact}

\usepackage{algorithm}
\usepackage[indLines=true]{algpseudocodex}
\usepackage{algorithmicx}
\algnewcommand\algorithmicinput{\textbf{Input:}}
\algnewcommand\Input{\item[\algorithmicinput]}
\algrenewcommand\algorithmicoutput{\textbf{Output:}}
\algrenewcommand\Output{\item[\algorithmicoutput]}
\algrenewcommand\algorithmicrequire{\textbf{Require:}}
\algrenewcommand\Require{\item[\algorithmicrequire]}

% --- CODE STUFF ---
\usepackage{minted}
\usemintedstyle{tango}
% from Stochastic Processes
\renewcommand{\Pr}[1]{\mathbf{Pr}\left[#1\right]}
\newcommand{\CPr}[2]{\mathbf{Pr}\left[#1\ |\ #2\right]}
\newcommand{\Ex}[1]{\mathbb{E}\left[#1\right]}
\newcommand{\ExSub}[2]{\mathbb{E}_#2\left[#1\right]}
\newcommand{\CEx}[2]{\mathbb{E}\left[#1\ |\ #2\right]}

\newcommand{\Exp}[1]{\text{Exp}\left(#1\right)}
\newcommand{\Pois}[1]{\text{Poisson}\left(#1\right)}
\newcommand{\DGamma}[2]{\text{Gamma}\left(#1, #2\right)}

% from Graph Theory
\DeclareMathOperator{\diam}{diam}

% misc. ones
\newenvironment{nexample}{
  \begin{leftbar}
    \noindent\textbf{Example.}
}{
  \end{leftbar}
}

\newenvironment{distraction}{
    \begin{tcolorbox}[title=Distraction!,breakable,colframe=red!69!black,before upper={\parindent15pt}]
}{
    \end{tcolorbox}
}

\newenvironment{hwproblem}[1]{\noindent\textbf{Problem #1.}}{}
\newmdenv[linewidth=0.5pt,linecolor=black,backgroundcolor=white]{singleframed}
\newenvironment*{singleframedindent}{
  \begin{singleframed}
    \setlength{\parindent}{\defparindent}\ignorespaces
  }{
  \end{singleframed}
}

\newcommand\sol[1]{
  \begin{singleframedindent}
    #1
  \end{singleframedindent}
}



\renewcommand{\questionlabel}{\textbf{Problem \thequestion.}}
\renewcommand{\solutiontitle}{}

\newcommand{\e}{\epsilon}
\newcommand{\tor}{\text{ or }}

\begin{document}

\section{Chapter 1}

\section{Chapter 2}


\begin{exercise}
In "$\epsilon$-greedy action selection, for the case of two actions and " $\epsilon= 0.5$, what is
the probability that the greedy action is selected?
\end{exercise}
\begin{solution}
Write $G:=\{\text{greedy action is selected}\}$. We have
\begin{align*}
Prob(\text{ G })=&Prob(\text{ G },random)+Prob(\text{ G },optimal)    \\
&=\epsilon+(1-\epsilon)\frac{1}{2}\\
&=0.75.
\end{align*}  
\end{solution}

%%%%%%%%%%%%%%%%%%%%%%%%%%%%%%%%%%%%%%%%%%%%%%%%%%%%%%%%%%%%%%%%%%%%%%%%%%%%%%%%%%%%%%%%%%%%%%%%%%%%%%%%%%%%%%%%%%%%%%%%%%%%%%%%%%%%%%%%%%%%%%%%%%%%%%%%%%%%%%%%%%%%%%%%%%%%%%%%%%%%%%%%%%%%%%%%%%%%%%%%%%%%%%%%%%%%%%%%%%%%%%%%%%%%%%%%%%%%%%%%%%%%%%%%%%%%%%%%%%%%%%%%%%%%%%%%%%%%%%%%%%%%%%%%%%%%%%%%%%%%%%%%%%%%%%%%%%%%%%%%%%%%%%%%%%%%%%%%%%%%%%%%%%%%%%%%%%%%%%%%%%%%%%%%%%%%%%%%%%%%%%%%%%%%%%%%%%%%%%%%%%%%%%%%%%%%%%%%%%%%%%%%%%%%%%%%%%%%%%%%%%%%%%%%%%%%%%%%%%%%%%%%%%%%%%%%%%%%%%%%%%%%%%%%%%%%%%%%%%%%%%%%%%%%%%%%%%%%%%%%%%%%%%%%%%%%%%%%%%%%%%%%%%%%%%%%%%%%%%%%%%%%%%%%%%%%%%%%%%%%%%%%%%%%%%%%%%%%%%%%%%%%%%%%%%%%%%%%%%%%%%%%%%%%%%%%%%%%%%%%%%%%%%%%%%%%%%%%%%%%%%%%%%%%%%%%%%%%%%%%%%%%%%%%%%%%%%%%%%%%%%%%%%%%%%%%%%%%%%%%%%%%%%%%%%%%%%%%%%%%%%%%%%%%%%%%%%%%%%%%%%%%%%%%%%%%%%%%%%%%%%%%%%%%%%%%%%%%%%%%%%%%%%%%%%%%%%%%%%%%%%%%%%%%%%%%%%%%%%%%%%%%%%%%%%%%%%%%%%%%%%%%%%%%%%%%%%%%%%%%%%%%%%%%%%%%%%%%%%%%%%%%%%%%%%%%%%%%%%%%%%%%%%%%%%%%%%%%%%%%%%%%%%%%%%%%%%%%%%%%%%%%%%%%%%%%%%%%%%%%%%%%%%%%%%%%%%%%%%%%%%%%%%%%%%%%%%%%%%%%%%%%%%%%%%%%%%%%%%%%%%%%%%%%%%%%%%%%%%%%%%%%%%%%%%%%%%%%%%%%%%%%%%%%%%%%%%%%%%%%%%%%%%%%%%%%%%%%%%%%%%%%%%%%%%%%%%%%%%%%%%%%%%%%%%%%%%%%%%%%%%%%%%%%%%%%%%%%%%%%%%%%%%%%%%%%%%%%%%%%%%%%%%%%%%%%%%%%%%%%%%%%%%%%%%
\begin{exercise}
Bandit example Consider a $k$-armed bandit problem with $k = 4$ actions, denoted $1, 2, 3$, and $4$. Consider applying to this problem a bandit algorithm using $\e$-greedy action selection, sample-average action-value estimates, and initial estimates of $Q_1(a) = 0$, for all $a$. Suppose the initial sequence of actions and rewards is $A1 = 1, R1 = -1, A2 = 2, R2 = 1, A3 = 2, R3 = -2, A4 = 2, R4 = 2, A5 = 3, R5 = 0.$ On some of these time steps the $\e$ case may have occurred, causing an action to be selected at random. On which time steps did this definitely occur? On which time steps could this possibly have occurred?
\end{exercise}
\begin{solution}


We denote the sample average action value after n-steps as
\begin{equation}
\Large{Q_t}(a) = \frac{{\sum\limits_{i = 1}^{t - 1} {{R_i}{1_{{A_i} = a}}} }}{{\sum\limits_{i = 1}^{t - 1} {{1_{{A_i} = a}}} }}.
\end{equation}

\begin{tabularx}{\linewidth}{ccccccc}
Timestep $t$ & $Q_{t+1}(1)$ & $Q_{t+1}(2)$    & $Q_{t+1}(3)$  & $Q_{t+1}(4)$ & Greedy action & Action selected\\
\hline 
t=0&0&0&0&0&-&$A_{1}=1$\\
\hline 
t=1&$\frac{R_{1}}{1_{{A_1} = 1}}=-1$  & 0 &0&0&$2,3\tor 4$&$A_{2}=2$\\
\hline 
t=2& $-1$ &1&0&0&2&$A_{3}=2$\\
\hline 
t=3& -1& $\frac{R_{2}+R_{3}}{2}=-0.5$ &0&0&3\tor 4&$A_{4}=2$\\
\hline 
t=4& -1 &$\frac{R_{2}+R_{3}+R_{4}}{3}=\frac{1}{3}$&0  &0&3&$A_{5}=3$\\
\hline 
t=5&-1&1/3&0&0&3&end\\
\hline
\end{tabularx}

\begin{enumerate}
    \item step $t=1$ can be either exploration or exploitation because the optimal value is zero for all actions.

    \item step $t=2$, could be either exploration or exploitation because action=2 is optimal too.

    \item step $t=3$, is exploitation because the only optimal action was 2.

\item step $t=4$, is exploration because the action 2 is not optimal.

\item step $t=5$, is exploitation because action=$3$ is optimal.
    
\end{enumerate}






\end{solution}


1)On which time steps did this definitely occur?\\
t=1 

t=4


2)On which time steps could this possibly have occurred?\\
t=2

t=5
because the optimal is action=2.

3)optimal selection\\

t=3 



%%%%%%%%%%%%%%%%%%%%%%%%%%%%%%%%%%%%%%%%%%%%%%%%%%%%%%%%%%%%%%%%%%%%%%%%%%%%%%%%%%%%%%%%%%%%%%%%%%%%%%%%%%%%%%%%%%%%%%%%%%%%%%%%%%%%%%%%%%%%%%%%%%%%%%%%%%%%%%%%%%%%%%%%%%%%%%%%%%%%%%%%%%%%%%%%%%%%%%%%%%%%%%%%%%%%%%%%%%%%%%%%%%%%%%%%%%%%%%%%%%%%%%%%%%%%%%%%%%%%%%%%%%%%%%%%%%%%%%%%%%%%%%%%%%%%%%%%%%%%%%%%%%%%%%%%%%%%%%%%%%%%%%%%%%%%%%%%%%%%%%%%%%%%%%%%%%%%%%%%%%%%%%%%%%%%%%%%%%%%%%%%%%%%%%%%%%%%%%%%%%%%%%%%%%%%%%%%%%%%%%%%%%%%%%%%%%%%%%%%%%%%%%%%%%%%%%%%%%%%%%%%%%%%%%%%%%%%%%%%%%%%%%%%%%%%%%%%%%%%%%%%%%%%%%%%%%%%%%%%%%%%%%%%%%%%%%%%%%%%%%%%%%%%%%%%%%%%%%%%%%%%%%%%%%%%%%%%%%%%%%%%%%%%%%%%%%%%%%%%%%%%%%%%%%%%%%%%%%%%%%%%%%%%%%%%%%%%%%%%%%%%%%%%%%%%%%%%%%%%%%%%%%%%%%%%%%%%%%%%%%%%%%%%%%%%%%%%%%%%%%%%%%%%%%%%%%%%%%%%%%%%%%%%%%%%%%%%%%%%%%%%%%%%%%%%%%%%%%%%%%%%%%%%%%%%%%%%%%%%%%%%%%%%%%%%%%%%%%%%%%%%%%%%%%%%%%%%%%%%%%%%%%%%%%%%%%%%%%%%%%%%%%%%%%%%%%%%%%%%%%%%%%%%%%%%%%%%%%%%%%%%%%%%%%%%%%%%%%%%%%%%%%%%%%%%%%%%%%%%%%%%%%%%%%%%%%%%%%%%%%%%%%%%%%%%%%%%%%%%%%%%%%%%%%%%%%%%%%%%%%%%%%%%%%%%%%%%%%%%%%%%%%%%%%%%%%%%%%%%%%%%%%%%%%%%%%%%%%%%%%%%%%%%%%%%%%%%%%%%%%%%%%%%%%%%%%%%%%%%%%%%%%%%%%%%%%%%%%%%%%%%%%%%%%%%%%%%%%%%%%%%%%%%%%%%%%%%%%%%%%%%%%%%%%%%%%%%%%%%%%%%%%%%%%%%%%%%%%%%%%%%%%%%%%%%%%%%%%%%%%%%%%%%%%%%%%%%%%%%%%%%%%%%%%
\subsection{2.3}
The epsilon=0.01 because the probability is $\frac{\sum 1_{A_{i}=a}}{N}\to \frac{\epsilon}{k}+1-\epsilon$. If we have epsilon=0, we get stuck in the case of $Q_1(a)=0$ to $A_{1}=0$.






%%%%%%%%%%%%%%%%%%%%%%%%%%%%%%%%%%%%%%%%%%%%%%%%%%%%%%%%%%%%%%%%%%%%%%%%%%%%%%%%%%%%%%%%%%%%%%%%%%%%%%%%%%%%%%%%%%%%%%%%%%%%%%%%%%%%%%%%%%%%%%%%%%%%%%%%%%%%%%%%%%%%%%%%%%%%%%%%%%%%%%%%%%%%%%%%%%%%%%%%%%%%%%%%%%%%%%%%%%%%%%%%%%%%%%%%%%%%%%%%%%%%%%%%%%%%%%%%%%%%%%%%%%%%%%%%%%%%%%%%%%%%%%%%%%%%%%%%%%%%%%%%%%%%%%%%%%%%%%%%%%%%%%%%%%%%%%%%%%%%%%%%%%%%%%%%%%%%%%%%%%%%%%%%%%%%%%%%%%%%%%%%%%%%%%%%%%%%%%%%%%%%%%%%%%%%%%%%%%%%%%%%%%%%%%%%%%%%%%%%%%%%%%%%%%%%%%%%%%%%%%%%%%%%%%%%%%%%%%%%%%%%%%%%%%%%%%%%%%%%%%%%%%%%%%%%%%%%%%%%%%%%%%%%%%%%%%%%%%%%%%%%%%%%%%%%%%%%%%%%%%%%%%%%%%%%%%%%%%%%%%%%%%%%%%%%%%%%%%%%%%%%%%%%%%%%%%%%%%%%%%%%%%%%%%%%%%%%%%%%%%%%%%%%%%%%%%%%%%%%%%%%%%%%%%%%%%%%%%%%%%%%%%%%%%%%%%%%%%%%%%%%%%%%%%%%%%%%%%%%%%%%%%%%%%%%%%%%%%%%%%%%%%%%%%%%%%%%%%%%%%%%%%%%%%%%%%%%%%%%%%%%%%%%%%%%%%%%%%%%%%%%%%%%%%%%%%%%%%%%%%%%%%%%%%%%%%%%%%%%%%%%%%%%%%%%%%%%%%%%%%%%%%%%%%%%%%%%%%%%%%%%%%%%%%%%%%%%%%%%%%%%%%%%%%%%%%%%%%%%%%%%%%%%%%%%%%%%%%%%%%%%%%%%%%%%%%%%%%%%%%%%%%%%%%%%%%%%%%%%%%%%%%%%%%%%%%%%%%%%%%%%%%%%%%%%%%%%%%%%%%%%%%%%%%%%%%%%%%%%%%%%%%%%%%%%%%%%%%%%%%%%%%%%%%%%%%%%%%%%%%%%%%%%%%%%%%%%%%%%%%%%%%%%%%%%%%%%%%%%%%%%%%%%%%%%%%%%%%%%%%%%%%%%%%%%%%%%%%%%%%%%%%%%%%%%%%%%%%%%%%%%%%%%%%%%%%%%%%%%%%%%%%%%%%%%%%%%%%%%%%%%%%%%%%

\subsection{}



\end{document}

